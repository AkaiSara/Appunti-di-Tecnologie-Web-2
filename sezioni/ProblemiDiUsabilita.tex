
\section{Problemi di usabilità}

	\subsection{Problemi persistenti}
		Sono i problemi gravi fortemente connessi alla tecnologia e che nel tempo non cambieranno.
	
		\subsubsection{Navigazione}
			Il problema del navigare nel web odierno è la possibilità di perdersi: \emph{lost in navigation}, ossia la prese di coscienza dell'utente che capisce di essersi perso. Fortunatamente, se opportunamente inserito, l'asse informativo Where risolve questo tipo di problema.
			\paragraph*{Morale:}
			\begin{quote}
				``Gli utenti devono essere coscienti di dove sono e dove devono andare."
			\end{quote}
			
			\paragraph{Dove sono, dove ero e dove sarò}
				Nonostante l'uso di \emph{breadcrump} e asse Where opportunamente comunicato all'utente ciò non basta. Può infatti capitare che l'utente si ritrovi in pagine già visitate e deva ricordarsi i percorsi già fatti. Lo sforzo diventa pesante e crea malumore. Per non far affaticare l'utente esiste al giorno d'oggi una consuetudine non standard riconosciuta in tutto il web che è quella di colorare diversamente i colori dei link già visitati. Ciò fu implementato da Netscape Navigator e da allora è diventata una buona norma per garantire maggior usabilità.
				Il 75\% dei siti web usa il cambio colore dei link già visitati.
			\subparagraph*{Morale:}
			\begin{quote}
				``All'utente pesa meno la grafica rispetto alla funzionalità e allo sforzo."
			\end{quote}	
			
		\subsubsection{I movimenti dell'utente}
			Le azioni generali per interagire che può utilizzare l'utente sono:
			\begin{itemize}
				\item il \emph{click}.
				\item il \emph{back} (pulsante prezioso, presto vedremo il perché).
			\end{itemize}
			Secondo gli studi sul comportamento degli utenti sul web si è scoperto che ad essi piace navigare all'indietro, anzi lo adorano. Prendiamo ad esempio la visita di un sito in cui si sia andati in profondità di 4 livelli e si deva tornare alla homepage. Gli utenti a questo punto spesso invece di cliccare una volta  il link diretto (magari sul logo del sito) preferiscono di gran lunga utilizzare il pulsante \emph{back} ripetutamente.
			Si arriva fino a 7 click del pulsante \emph{back} anche in presenza di un link diretto. È lo stesso comportamento che si tiene con il telecomando della propria TV. A volte basterebbe premere i pulsanti numerici per passare ad un diverso canale ma si preferisce spostarsi di un canale alla volta usando un unico bottone invece di due o più bottoni numerici. Questo uso comune è noto come \emph{backtracking}.
			\subparagraph*{Morale:}
			\begin{quote}
				``La pulsione primaria dell'utente non è quella di minimizzare il tempo ma quella di minimizzare lo sforzo."
			\end{quote}
			L'uomo ha orrore nello sforzo previsto nel futuro e tende a fare cose folli e illogiche per allontanare tale sforzo (si veda \emph{l'algoritmo della carta igienica}). Quindi, gli utenti minimizzano lo sforzo computazionale e per fare ciò ricorrono all'uso del pulsante \emph{back} perché:
			\begin{itemize}
				\item non serve ricordarsi il percorso;
				\item non bisogna trovare il tasto \emph{back}, (è sempre lì garantito).
			\end{itemize}
			Da ciò ovviamente ne consegue che non bisogna \textbf{mai togliere l'uso del \emph{back button}}
			
		\subsubsection{Nuova finestra? No, grazie}
			Un altro problema persistente è quello di aprire una nuova finestra di navigazione anziché usare sempre la stessa.  Esistono due tipi di finestre, il tab e la nuova finestra vera e propria. L'aprire una nuova finestra ha gravi conseguenze per l'utente medio:
			\begin{itemize}
				\item Non c'è più la cronologia di navigazione (addio \emph{back button}!
				\item Avere finestre diverse aperte confonde l'utente medio.
			\end{itemize}
			Analiziamo nel dettaglio che cosa susccede all'apertura di una nuova finestre. Prima di tutto questa si sovrappone alla navigazione esistente provocando panico per l'utente medio. Se dovesse non sovrapporsi l'utente medio seleziona quella bassa lasciando l'altra finestra aperta. Di conseguenza il link della pagina già aperta non funziona più perché la pagina risulta aperta ma di ciò l'utente medio non ne è a conoscenza.
			
			\paragraph{Un problema correlato: i pop-up}
				Un problema che si collega molto con l'apertura di una nuova finestra è quello dei pop-up. Piccole finestre che si aprono senza il permesso dell'utente (si veda in seguito per maggiori dettagli).
		
		\subsubsection{Convenzioni violate}
			Le convenzioni non sono standard ma semplicemente la prassi, ciò che fanno tutti e per questo più familiari all'utente. 
			
			\paragraph{Legge di Jacob}
			\begin{quote}
				``Gli utenti spedono la maggior parte del tempo su altri siti web."
			\end{quote}
			Gli utenti sono abituati a navigare in altri siti quindi non abbiamo il potere di fare tutto di testa nostra solo perché è il "nostro" sito.
			
		\subsubsection{Altri problemi: What non rispettato}
			Mai usare linguaggio vuoto o con poco contenuto/slogan. L'uente che visita una pagina si aspetta contenuto non ``politichese" cit.
			
			\paragraph{Problema correlato: la forma del testo}
				Il contenuto di una pagina web conta ma il testo deve sempre avere una forma semplice, chiara e sintetica. Mai usare testo difficile e monolitico che spesso, purtroppo, è usato nei siti delle pubblica amministrazione. Il testo usato su altri media non è adatto al web. Alcuni accorgimenti per evitare ciò è quello di tagliare testo.
				\begin{itemize}
					\item Se abbiamo del normale testo da inserire in una pagina bisogna \textbf{dimezzare} per far sì che diventi testo web.
					\item Se abbiamo testo generico, il testo web è circa \textbf{un quarto}.
				\end{itemize}
				Un altro suggerimento per scrivere testo adatto al web è quello di cominciare con la conclusione e successivamente espandere.
	
	\subsection{Problemi non-persistenti}
	
		\subsubsection{Splash page}
		
		\subsubsection{Richieste di registrazione}
		
		\subsubsection{Lo scrolling maledetto}
			\paragraph{Scrolling verticale}
			\paragraph{Taglia dello schermo}
			\paragraph{Scrolling orizzontale}
		
		\subsubsection{Bloated design}
		
		\subsubsection{Abusi multimediali}
			\paragraph{Il 3D - Prima, dopo, ora}
			\paragraph{Il plugin}
			\paragraph{Flash!}
			\paragraph{I video}
		
		\subsubsection{Metafora visiva}
		
		\subsubsection{I menu di navigazione}
			\paragraph{Pathfinding}
			\paragraph{Fault-tollerant}
			
		\subsubsection{Il testo}
			
			\paragraph{Caps lock}
			\paragraph{Immagini sostitutive}
			\paragraph{La maledizione Lorem Ipsum}
				\subparagraph{L'effetto ghigliottina}
				
		\subsubsection{Scanning}
			\paragraph{Strutturazione}
			\paragraph{Problemi}
			\paragraph{Blonde effect}		