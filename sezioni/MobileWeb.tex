
\section{Mobile Web (e App)}
	Solo recentemente i dispositivi mobile hanno spopolato è la tecnologia ha superato di gran lunga i web designer che sono impreparati nell'ambito mobile emergente. Nel 2013 l'accesso a internet da dispositivi mobile ha superato quello di desktop e laptop ed è in costante crescita. Nonostante questo trend, 530 siti nella top 1000 del mondo non dispongono di una versione mobile e il 25\% di questi sfora lo schermo.

	\subsection{Un po' di storia}
		Nel marzo 2013 avviene un importate scelta aziendale in Google. Il team di sviluppo di Android che non portava risultati soddisfacenti è inglobato dal team di Chrome che invece aveva successo. L'idea era ed è quella di avere convergenza tra mondo mobile e web. 
		
		Già prima si era cercato di percorrere questa rotta da Google ma le cose non andarono bene visti i contrasti con Apple che non voleva collaborare. Si pensi che lo stesso Steve Jobs era contrario alle app. Da qui il motivo dell'acquisto del sistema Android da parte di Google.
		
		Ora il percorso è ben delineato: la nascita delle hybrid apps scritte usando HTML5 e multipiattaforma segnano ancora più visivamente la convergenza tra mobile e web.
	
	\subsection{Le App}
		Nascono dalla necessità di minimizzare ancora una volta lo sforzo computazionale delle persone. L'app minimizza enormemente il tempo di accesso al servizio richiesto dagli utenti. Di conseguenza questo porta a maggiori esigenze da parte degli utenti e riduce i timer di soddisfazioni.
		
		\subsubsection{Parlano le statistiche}
			Dalle statistiche emerge:
			\begin{itemize}
				\item quasi un quarto degli utenti usano app più di 60 volte al giorno 
				\item e questo cresce ogni anno del 123\%!
				\item La fascia d'età che meno 'drogata' di app si trova tra i 25 e 35 anni (i motivi sembrano principalemente per la mancanza di tempo).
			\end{itemize}

			Le app vincono sul mobile web, gli utenti smartphone passano in media l'84\% di tempo giornaliero sulle app e solo il 14\% sul web vero e proprio. Nella pratica si capisce il perché:
			\begin{itemize}
				\item il 32\% di questo tempo è speso in \textbf{giochi} (non sorprende quindi la scelta del nome Google Play per lo store di Google).
				\item il 28\% sui \textbf{social}, il 17\% è Facebook!
			\end{itemize}
			Da notare che tutto questo uso di app (giochi a parte) è solo fruizione di contenuti nel web tramite l'app apposita.
			
		\subsubsection{L'arena delle App}
			Quando le statistiche parlano chiaro e muovono un sacco di persone si muovono anche un sacco di soldi e ricerca di successo. È per questo motivo che nel mercato delle app, attualmente, c'è un'enorme competizione:
			\begin{itemize}
				\item Le app hanno vita media bassissima: dai \textbf{4 mesi} ad \textbf{1 anno}.
					\begin{itemize}
						\item i game hanno vita media di soli \textbf{4 mesi}.
					\end{itemize}
				\item Se un app resiste ed è ancora in crescita dopo 3 mesi avrà una vita lunga altrimenti è defunta e da considerarsi un insuccesso.
			\end{itemize}
			
			\paragraph{La sequenza della morte}
				Di seguito quella che viene chiamata la \emph{sequenza della morte} di una app descrive al meglio quello già descritto sopra, riportano i dati del comportamento degli utenti di fronte ad un'app.
				\begin{itemize}
					\item il \textbf{26\%} delle app è aperta al massimo \textbf{una volta}.
					\item il \textbf{13\%} sono aperte al massimo \textbf{2 volte}.
					\item il \textbf{9\%} sono aperte al massimo \textbf{3 volte}.
					\item il \textbf{50\%} degli utenti apre le app al massimo 3 volte e poi 
				\end{itemize}
		
		\subsubsection{Alla ricerca dell'App}
			Tanta competizione e tante app defunte in pochissimo tempo. Ma come trovare queste app? È qui che il paragone con i siti web è possibile. Come per essi esistono i motori di ricerca anche per le app esistono questi: gli store. Anche qui infatti si presenta il problema di essere trovati ai primi posti della ricerca nello store proprio come per i siti internet. Per fare ciò è nata l'ASO.
			
			\paragraph{ASO: App search optimization}
				È il corrispondente CEO per le app e presenta di fatto delle somiglianze prima su tutte funziona per \emph{keywords} che richiede quindi sforzo per un'\textbf{ottimizzazione testuale} sui pochi luoghi disponibili nello store.
				\begin{itemize}
					\item Descrizione app.
					\item Spazio apposito per le keyword.
					\item Nome dell'app (corrisponde al nome del sito vedere indice NOMI).
				\end{itemize}
				Poichè non si possono utilizzare tecniche ipertestuali i motori di ricerca degli store applicano l'uso dei dati del sistema sociale complessivo (SIS) che si basa su quanto segue:
				\begin{itemize}
					\item numero di download (integrati nel tempo).
					\item tempo d'uso dell'app.
					\item \emph{ratings} e \emph{review}.
					\item disinstallazioni.
					\item brand.
					\item metriche di motori di ricerca del web. Per esempio su Google Play sono integrate tutte le metriche positive e negative raccolte sul web per quell'app.
				\end{itemize}
							
			
	\subsection{Usabilità: mobile e desktop}
		Per valutare se una pagina è corretta per dispositivi mobile esistono potenti strumenti. Prima fra tutti il \emph{Google mobile compatibility test}. Esso verifica che siano rispettate alcune caratteristiche che possiamo catalogare in tre componenti base:
		\begin{enumerate}
			\item Essere mobile.
			\item Taglia dello schermo.
			\item Interazione.
		\end{enumerate}
		
		\subsubsection{L'esempio di Facebook}
			Una considerazione è doverosa farla sui diversi tipi di device oggi in commercio. Oltre a diverse composizioni hardware abbiamo diverse funzionalità offerte dagli telefoni cellulari. Bisogna porre attenzione al target di riferimento, si pensi ad esempio che non tutti i telofoni hanno il touch.
			L'esempio del social network mondiale Facebook è esplicativo del problema. Facebook per risolvere questi problemi infatti offre addirittura 3 versioni mobile del sito \emph{facebook.com}:
			\begin{description}
				\item [m.facebook:] versione per cellulari non touch.
				\item [touch.facebook:] versione per cellulari touch.
				\item [0.facebook:] versione a banda ultra ridotta offerto gratuitamente in tutte le zone dove le reti telefoniche sono lente (fidelizzazione globale dei clienti).		
			\end{description}
		
		\subsubsection{Le tre caratteristiche del mobile}
			
			\paragraph{Essere Mobile}
			\paragraph{Taglia dello schermo}
				\subparagraph{Invasività}
			\paragraph{Le dita}
				\subparagraph{Reversibility principle}
				\subparagraph{Fitts, il ritorno}