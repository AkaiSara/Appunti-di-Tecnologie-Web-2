
\section{Il comportamento degli utenti}
	Nel 1990 e 1991 si sono effettuati diversi sudi sul comportamento degli utenti con metodologie varie. Da questi è sorto che per ogni tipo di media l'uomo mette in atto delle regole ben precise. 
	Vedremo un esempio di media classico e il perché le sue regole \textbf{non} sono da riutilizzare nel web, quali sono le differenze e quali strumenti abbiamo a disposizione per diminuire gli sforzi dell'utenza.
	
	\subsection{Un media classico: Il giornale}
		
		\subsubsection{L'importanza delle immagini}
		
	\subsection{Il web}
		
		\subsubsection{Fase di attenzione}
		
		\subsubsection{L'importanza del testo}
		
			\paragraph{Paragrafi e titoli}
			
		\subsubsection{Pagine grasse o magre?}
		
		\subsubsection{Immagini nel web}
			
			\paragraph{Slideshow}
			
		\subsubsection{Lo spostamento dell'utente}
		
			\paragraph{Past\&clink o drag\&drop?}
		
			\paragraph{Legge di Fitts}
			
				\subparagraph{Implicazioni della legge di Fitts}
				\subparagraph{Complicazioni}
				
				\subparagraph{Bordi e angoli}
				
				\subparagraph{Menu 2.0}
				
				